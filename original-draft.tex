% Created 2021-10-02 Sat 13:15
% Intended LaTeX compiler: pdflatex
\documentclass[11pt]{article}
\usepackage[utf8]{inputenc}
\usepackage[T1]{fontenc}
\usepackage{graphicx}
\usepackage{grffile}
\usepackage{longtable}
\usepackage{wrapfig}
\usepackage{rotating}
\usepackage[normalem]{ulem}
\usepackage{amsmath}
\usepackage{textcomp}
\usepackage{amssymb}
\usepackage{capt-of}
\usepackage{hyperref}
\usepackage{mhchem}
\usepackage{svg,mathptmx}
\author{Alyaman Maasarani ID:202203638 \\ \\
  \small{ Email: ahm363@student.bau.edu.lb } \\
\small{Team Members: Abdalkader, Khaleel}}
\date{27 September, 2021}
\title{Experiment 3 Report \\ (Qualitative Analysis of Cations) \\ \small{ CHEM241L-T1 -
  Principles of Chemistry Lab }}
\hypersetup{
 pdfauthor={Alyaman Maasarani},
 pdftitle={Experiment 3 Report (Qualitative Analysis of Cations)},
 pdfkeywords={},
 pdfsubject={},
 pdfcreator={Emacs 28.0.50 (Org mode 9.4.6)}, 
 pdflang={English}}
\begin{document}

\maketitle
\tableofcontents


\section{Purpose}
\label{sec:org33dd161}
The purpose of this experiment is to learn how to detect the presence/absence of certain cations through their reactions with certain substances. At the end, the student is able to detect \(Ag^{+}\), \(Ba^{2+}\) and \(Fe^{3+}\). Additionally, we also learn how to properly separate precipitate from liquid, how to filter a liquid, and how to clean glassware.

\section{Scheme}
\label{sec:org9afc1ef}
\begin{center}
  \makebox[0pt]{\scalebox{0.8}{\includesvg{scheme.svg}}}
\end{center}


\section{Equations}
\begin{align}
  \ce{Ag^{+}(aq) + Cl^{-}(aq) &<=> AgCl(s)} \\
  \ce{AgCl(s) + 2NH3(aq) &<=> [Ag(NH3)2]^{+}(aq) + Cl-(aq)}\\
  \ce{[Ag(NH3)2]+(aq) + Cl^{-}(aq) + 2H^{+}(aq) &<=> AgCl(s) + 2NH4^{+}(aq) }\\
  \ce{ Fe3+(aq) + 3NH3(aq) + 3H2O(l) &-> 3NH4+(aq) + Fe(OH)3(s) } \\
  \ce{3HCl + Fe(OH)3 &-> FeCl3 + 3 H2O}\\
  \ce{FeCl3 + 3NH3 + 3H2O &-> Fe(OH)3 + 3NH4Cl}\\
  \ce{4Fe^{3+}(aq) + 3K4[Fe(CN)6](aq) &<=> Fe4[Fe(CN)6]3(s, blue) + 12K+}\\
  \ce{FeCl3 + 3KSCN &-> Fe(SCN)3 + 3KCl}\\
  \ce{BaCl2(aq) + K2CrO4(aq) &-> BaCrO4(s) + 2KCl(aq)}
\end{align}

\section{Conclusion}
\label{sec:orgf90cc25}
We conclude that \(Ag^+\) turns into a white precipitate with \(HCl\), and returns into a precipitate after adding \(NH_3\) and \(HNO_3\) with release of heat as a sign. \(Fe^{3+}\) reacts with \(NH_3\) and turns a red precipitate scattered around the tube, and can further turn into a red or dark blue solution if we add \(KSCN\) or \(NH_3 + K_4[Fe(CN)_6]\) respectively. \(Ba^{2+}\) turns into a precipitate with a light yellow solution upon addition of \(K_2CrO_4\). These reactions allow us to detect the presence of \(Ag^+\), \(Fe^{3+}\) and \(Ba^{2+}\).
\end{document}