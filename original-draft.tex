% Created 2021-09-28 Tue 16:44
% Intended LaTeX compiler: pdflatex
\documentclass[11pt]{article}
\usepackage[utf8]{inputenc}
\usepackage[T1]{fontenc}
\usepackage{graphicx}
\usepackage{grffile}
\usepackage{longtable}
\usepackage{wrapfig}
\usepackage{rotating}
\usepackage[normalem]{ulem}
\usepackage{amsmath}
\usepackage{textcomp}
\usepackage{amssymb}
\usepackage{capt-of}
\usepackage{hyperref}
\usepackage{svg}
\usepackage{mathptmx}
\author{Alyaman, Abd Alkader, Khaleel}
\date{\today}
\title{Experiment 3 Observations Draft}
\hypersetup{
 pdfauthor={Alyaman, Abd Alkader, Khaleel},
 pdftitle={Experiment 3 Observations Draft},
 pdfkeywords={},
 pdfsubject={},
 pdfcreator={Emacs 28.0.50 (Org mode 9.4.6)}, 
 pdflang={English}}
\begin{document}

\maketitle
\tableofcontents


\section{Original Experiment (\(Ag^+ Ba^{2+} Fe^{3+}\) Solution)}
\label{sec:org98f0e8f}
\subsection{Testing for \(Ag^+\)}
\label{sec:org412e963}
We first added 1 dropper of \(HCl\) to the solution, and observed a white precipitate forming at the bottom. This signifies the potential existence of \(Ag^+\).
We separate this precipitate into a new tube, and keep the supernatant solution in the original tube.
After adding Ammonia to the precipitate (in the new tube), and stirring so it dissolves, the color turns teal-like\footnote{After adding an extra dropper of Ammonia, it then turned yellow.}.
We then added 1 dropper of \(HNO_3\), the solution quickly turned white. Also there's a release of heat. There's a precipitate left.
This means that \(Ag^+\) exists in the original mixture.
\subsection{Testing for \(Fe^{3+}\)}
\label{sec:orga95bc3e}
We return back to the original tube, and add Ammonia. We notice the appearance of a red/orange layer on top\footnote{We had to add 3 droppers of Ammonia for the layer to persist.}. This signifies the potential existence of \(Fe^{3+}\).
After pouring the contents of the tube on a filter directed at a new tube A, the red color is filtered and does not fall down into the tube A. Only the supernatant solution passes through the filter and into the new tube A\footnote{Tube A now contains the rest of the original mixture, which now only contains \(Ba^{2+}\).}.
We put the filter on another new tube B, and added \(HCl\) into it, we observed the release of smoke and the solution starts turning yellow and starts to go through the filter. The Tube gets blurry.
We divided this solution in tube B into 2 tubes, tube 1 and tube 2.
\subsubsection{Tube 1}
\label{sec:org5e37295}
We added 3 drops of \(KSCN\), a dark red/black color appears on top, fading into the yellow at the bottom. The whole tube turns red after a while. This signifies the existence of \(Fe^{3+}\).
\subsubsection{Tube 2}
\label{sec:org907daeb}
We added 1 drop of Ammonia, with no significant change\footnote{Though the tube blurs a bit right above the solution. This probably signifies the release of heat.}.
We then added Cyanide (\(Ku\)), and observed the solution turns dark blue after mixing/stirring, this signifies the existence of \(Fe^{3+}\).
\subsection{Testing for \(Ba^{2+}\)}
\label{sec:org1ea1dab}
We added 4 drops of Potassium Chromate to the tube A, which now contains \(Ba^{2+}\). The color slowly turns light yellow with visible white in the mix. A precipitate starts forming. This signifies the existence of \(Ba^{2+}\).
\subsection{Graph/Scheme:}
\label{sec:org9a03ddc}
\begin{center}
  \makebox[0pt]{ \scalebox{0.85}{ \includesvg{scheme.svg} } }
\end{center}
% \begin{figure}
%   \centering
%   % \def\svgwidth{\columnwidth}
%   \input{scheme.pdf_tex}
% \end{figure}
\section{Unknown 1}
\label{sec:org4bc42f6}
\subsection{Testing for \(Ag^+\)}
\label{sec:org82ebfb8}
With \(HCl\): White precipitate and yellow solution --> \(AgCl\) --> presence of \(Ag\).
We add \(NH_3\) into the tube with the precipitate: No reaction observed.
\subsection{Testing for \(Fe^{3+}\)}
\label{sec:org6b42058}
We add 4 droppers of \(NH_3\) to the original tube (with the supernatant solution), a red precipitate forms. So \(Fe^{3+}\) exists. We separate this red layer from the solution with a filter.
\subsection{Testing for \(Ba^{2+}\)}
\label{sec:orgc9adc99}
We add \(K_2CrO_4\), and observe no reaction of formation of precipitate. So Barium does not exits.
\section{Unknown 3}
\label{sec:orgc92c917}
\subsection{Testing for \(Ag^+\)}
\label{sec:org2cdec9a}
We add 1 dropper of \(HCl\) to the original solution, the solution turns yellow with no formation of a precipitate. So no \(Ag\).
\subsection{Testing for \(Fe^{3+}\)}
\label{sec:orgffcbb72}
We add 2 droppers of Ammonia, and observe a red/yellow color throughout the mixture. So \(Fe^{3+}\) exists. We filter this layer into another tube.
\subsection{Testing for \(Ba^{2+}\)}
\label{sec:orgdc6b925}
We add a few drops of Potassium Chromate to the original tube with no \(Fe^{3+}\), a precipitate appears and the color turns light yellow/white. So Barium exists.
\end{document}